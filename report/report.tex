\documentclass[a4wide, 11pt]{article} \usepackage{a4, fullpage}
\setlength{\parskip}{0.3cm} \setlength{\parindent}{0cm}

\usepackage{graphicx}
\usepackage{float}
\usepackage[top=0.75in, bottom=0.75in, left=0.75in, right=0.75in]{geometry}
\usepackage{array,booktabs}

% Stop Latex from repositioning tables like an idiot
\restylefloat{table}

% Itemised columns
\makeatletter
\newcolumntype{i}[1]{%
    >{\minipage[t]{\linewidth}\let\\\tabularnewline
      \itemize
      \addtolength{\rightskip}{0pt plus 50pt}% for raggedright
      \setlength{\itemsep}{-\parsep}}%
    p{#1}%
    <{\@finalstrut\@arstrutbox\enditemize\endminipage}}
\makeatother

\begin{document}

\begin{center} {\huge 3rd Year Group Project Final Report} \\ [0.4cm]
{\large Sean Allan, Mitchell Allison, Sam Esgate, Tom Harling, Ted Sales, Max Tottenham} \\ [0.2cm]
\vspace{0cm} \end{center}

\section{Executive Summary}

Our project at it's heart is a language translator, it translates a restricted
version of First Order Predicate Logic into the database querying language
SQL. This allows us to open up new avenues in the realms of teaching, database
administration, and computer security.

In the world of computer security, access to data is an important topic. 
One which has come to public attention through the recent leaks from NSA
contracter Edward Snowden. Our product provides the bedrock for an information 
security system. 

Imagine a scenario where your company needs to work with another large organisation, 
for example a Telecomunicaitons company and the United States Governement. Your company
is passionate about your customers privacy rights and so you want to give the 
government as little information as possible about your customers, whilst still alowing
them to do their job. This could be as secretive as not even allowing the governement 
to know what kinds of data you hold. What you would like is a system which would allow
the goverment to ask you a question, and for you to be able to respond with as little 
information as possible to answer that question without the government konwing what 
information you hold. 

Our product can provide the bedrock for this. Questions can be formed in predicate 
logic about whether a particular entity exists and if certain conditions surounding 
it hold, our project then takes this query and translates it into SQL, so that it is
ready to be sent to a conventional database.

From the teaching standpoint 


        Your elevator pitch


        What is your Project? What does it do? Why would I want to buy it? etc.
        No implementation, software engineering details, or project management

\section{Introduction}

        Set the scene ... motivation'
        State the problem you are trying to solve ...objective(s)
        Summarise your main achievements 

        -- GOALS - In Chatley coursework 1 --
        TODO: Discuss goals, they're referenced in the Project Management
        section. Revisions should not be talked about here though, just the
        overarching goals of the project.

\section{Design and Implementation}

        Detail your design (why did you do it this way?) - design of your software, possibly including a diagram of the major components
        Summarise key implementation details (how did you do it? what technology was used and why? what other technology was considered, but not used and why?
        Any technical challenges encountered and how addressed?
        Any risks anticipated, and how mitigated 

\section{Evaluation}

        Evaluate your deliverables e.g. performance, usability, etc.
        Summarise testing procedures + relevant testing results 

\section{Conclusion and Future Extensions}

        What did you learn? What might you have done differently?
        How would you build on what you have done? 

\section{Project Management}

        Planning, group organisation, breakdown + task allocation etc.

\subsection{Planning}

-- TODO Should we discuss here how everything went wrong, and revisions were
missed entirely? --

As we have learnt from previous projects, inside and outside the department, it
is critical to thoroughly plan a software engineering exercise of this scale
before beginning to implement features and write code. As discussed in our
introduction section, we decomposed our problem of translating first order
predicate logic to SQL into several key subgoals. This enabled us to outline a
core feature set which we wished to implement and gave us a much clearer idea
of how to begin tackling the problem.

Moreover, we initially split the project up in to five major 'revisions'. Each
revision added a certain amount of functionality to our project, and allowed us
to continually build upon features already implemented. This incremental method
of development is one of the key ideas of the Agile development methodology to
which we adhered to throughout the project. -- TODO citation -- Additionally,
with each revision having a deadline, it gave the team a clear plan of what was
to be implemented at what time.

\begin{table}[H]
  \centering
  \begin{tabular}{| l | i{0.6\textwidth} | l |}
    \hline
    \textbf{Revision} & \multicolumn{1}{p{0.6\textwidth} |}{\textbf{Necessary Steps for
Completion}} & \textbf{Completion Estimate} \\
    \hline
    1 & \item Set up basic web server;
        \item Set up and create a database;
        \item Create simple UI \& communicate with server;
        \item Create 5 sample Logic to SQL translations;
        \item Set up development environment.
    & 17/10/13 \\
    \hline
    2 & \item UI sends predicate logic to web server;
        \item Web server parses the logic into an AST;
        \item Backend translates the AST to SQL for basic “SELECT FROM” queries (i.e.
              projection only);
        \item Create 5 more sample Logic to SQL translations.
    & 25/10/13 \\
    \hline
    3 & \item Expand backend parser grammar to include more advanced use of SQL
              queries (selection \& projection);
        \item Dynamic table selection in logic - "smart" logic predicates, i.e.
              updated(x) vs. customer\_updated(x);
        \item Hook into database.
    & 1/11/13 \\
    \hline
    4 & \item User-defined functions;
        \item Configurable UI - database settings, and possibly changing the UI depending
              on the user's ability;
        \item Extend backend to use SQL joins.
    & 8/11/13 \\
    \hline
    5 & \item Contingency time built in for bug fixes, or additional
    features/extensions depending on the project status (e.g. Semantic
    checks of logical statements before translating to SQL). & 15/11/13 \\
    \hline
  \end{tabular}
  \caption{Feature sets for each revision, as defined at the start of the
project}
\end{table}

\subsection{Group Organisation}

One of the key ways we kept our group organised was through weekly meetings.
During each weekly meeting, we discussed what tasks were to be completed that
week, and spoke in detail about any concerns members had about the tasks that
had been completed previously. We would also analyse the previous weeks
performance, and make changes to our plan accordingly. An example of such a
change is when we realised implementing the parser would take significantly
longer than originally thought. This lead us to modify our feature set,
reallocate group roles temporarily, and modify deadlines and revision subgoals.


\subsection{Tracking}

-- Agile manifesto, responding to change
-- Quantitatively and quantitively identify performance

-- Trello
-- Burndown chart

\section{Bibliography}

-- TODO Agree on referencing practice. The library recommends Harvard. --

https://workspace.imperial.ac.uk/library/Public/Harvard\_referencing.pdf

\section{Appendix}

The appendix is optional, and does not count towards the 35 pages. It may contain thing like: User guide, installation instructions; more extensive design, testing, statistics etc.

\section{Anandha waffle from his site - delete once finished report}

Final Report ? due: 13th Jan 2014, at 16:00 (both Electronic and Hardcopy)

Contents for Final Report: The project report should not be longer than 35 pages (recommended length is around 30 pages), and might be organised according to the following structure: see above sections

Make sure that the final report presents a coherent story. Ask advice from your supervisor. You might also draw inspiration from the instructions about writing up your individual project.

Bear in mind, that most of the project assessors will not have followed the project throughout and will only have a short time to listen to a presentation or see a demonstration. For this reason they will rely heavily on the report to judge the project.

The report should be submitted to SGO in form of a hard copy, as well as electronically through CATE. i.e. report.pdf. 

------------------------

Assessment
Will be updated soon.

Group project
The group project assessment is undertaken by each group's supervisor, and moderated by a larger group of assessors, who will attend your presentation, and/or read your final report. The assessment is based on:

    Executive Summary, 5
    Presentation, 10
    Group Collaboration and Management, 20
    Report, 30
    Technical Achievement, 35

The group project (along with the Software Engineering course) is worth 440 Marks for both the MEng and BEng students.
Overall assessment
The overall assessment is the sum of the group project component and the Software Engineering Course in an 80:20 split. 

\end{document}
